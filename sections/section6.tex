\section{ICT for Behavior Change}
Peoples everyday decisions largely impact their energy consumption.
Consider for example how we heat our households and choose to travel.\\

Finding out which behavioral interventions are effective are typically achieved through
\begin{itemize}
    \item Qualitative interviews
    \item Surveys
    \item Experiments
\end{itemize}
Experiments have earlier been very costly, but ICT enables cheaper to perform large scale experiments.

\subsection{Motivations behind Behavioral Change}
Research shows that people often hold incorrect beliefs about what motivates them, resulting in surveys being inaccurate.
People are even sometimes unable to identify true causes of past behavior.
Normative information (social pressure) has shown to be highly effective in motivating behavioral change.

\subsubsection{Boomerang Effect}
When people are compared to averages households under the average are observed to increase consumption and household over the average decrease their consumption.
This results in a zero-sum boomerang effect.
A solution to this is to enforce an \textbf{injunctive} social norm (judgement based on behavior relative to norm) instead of a \textbf{descriptive} social norm (do what others do).

\begin{tcolorbox}
\textbf{Case Study; Amphiro Smart Shower Meter}\\
Amphiro is a shower meter that attaches to the tubing and measures water volume, energy consumption and temperature.\\

Treatment group started taking shorter showers and got a better perception of their water usage.
The energy saved by applying feedback directly connected to the activity (showering) was much higher than any savings observed when energy part of aggregated electricity use.
The meters were also deployed in hotels, showing similair results.
Results are therefore not completely based on participants knowing that they are part of a study.

\end{tcolorbox}

