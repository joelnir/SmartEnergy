\section{Electrical Vehicles}

\subsection{Transportation}
A third of the total energy use of the EU goes into transport.
Road transport is the main reason behind emissions.
Transport has been increasing and continues to increase.
In the US and Europe petroleum is the main energy source for transportation.\\

Measuring the efficiency can be done in multiple ways.

\begin{itemize}
    \item Energy consumption / distance traveled
    \item Emissions / distance traveled
    \item Emissions / person-kilometer (pkm)
    \item Emissions / ton and km (for cargo)
\end{itemize}

\subsection{Motivation behind Electrical Vehicles (EVs)}
The demand for cars is increasing and expected to triple in the next 30 years.
The main growth is in developing countries.
On the other hand the oil supply is very limited and expected to decline.
This does not add up and major changes in the car industry are likely.
Many countries have set ambitious goals for the development of EVs.

\subsubsection{Regenerative Breaking}
When slowing down in traffic kinetic energy is wasted.
By regenerative breaking an electric engine can be used as a generator and recover some of breaking energy.
Regenerative breaking typically only requires a small engine and battery.

\subsection{Current state of EV fleet}
In most countries electrical vehicles and hybrids represent a tiny part of the countries car fleet.
Registrations of EVs are however increasing.
An exception is Norway, where EVs and hybrids represent 51\% of new car sales.
This early adoption can be attributed to strong political incentives providing financial subsidies for EVs.

\subsection{Challenges}

\begin{labeling}{\textbf{Battery conflicts}}
    \item [\textbf{Cost}]
    The main cost of EVs has been batteries.
    Battery costs are however currently coming down quickly.
    \item [\textbf{Battery safety}]
    The combination of high energy and flammable materials make battery pack safety an important concern.
    A number of safety tests have to be passed before batteries are certified safe.
    \item [\textbf{Battery lifetime}]
    We typically expect to use cars for a much longer timespan than other battery powered devices.
    \item [\textbf{Battery conflicts}]
    Requirements on battery power, lifetime, recharge time and cost often conflict.
    Battery technology that handles these conflicts exists, but is still expensive.
    The most commonly used type today is Li-ion.
    \item [\textbf{Driving range}]
    The driving range of EVs has increased a lot in the last years.
    In reality the vast majority of trips are very short.
    Statistically almost all trips can be covered by the driving range of modern EVs.
\end{labeling}

\subsection{Types of EVs}
Electrical vehicles can be split into three groups.

\subsubsection{Battery Electric Vehicles (BEV)}
BEVs are only electric and therefore have a comparably limited driving range. They take multiple hours to charge.

\subsubsection{Plug-in hybrid EV (PHEV)}
PHEVs use regenerative breaking as well as plug charging.
They reach higher driving ranges as they can utilize the gasoline engine.
The battery is typicaly smaller than in BEVs and charging thereby faster.

\subsubsection{Hybrid Electric Vehicle (HEV)}
HEVs are only equipped with regenerative braking.
They are generally similar to gasoline-powered cars.\\

PHEVs and HEVs can act as a bridging technology for a transition to EVs.
These types of EVs relieve any range anxiety of consumers.
More people using hybrids creates motivation for investing in network of charging places.
This in turn make more people consider BEVs as a viable option.

\subsection{Effect of EVs on the Grid}
Replacement of all cars with EVs would nearly double the electricity consumption of households.
This creates a major concern for the grid.
EV charging can be considered in multiple ways

\begin{itemize}
    \item Home charging
    \item Public charging
    \item Semi-public charging (e.g. at work)
    \item Special operation centers
\end{itemize}

If all EVs are charged at home this would create massive peaks as people come home from work in the evening.
Similar effects could be expected if everyone charged at work, since most people arrive to work at similar times.
Load management that spreads out the charging over time can help relieve these issues somewhat.
Even with load management cars moving in clusters between the city during the day and suburbs during the evening can be a challenge for the grid.



