\section{World Energy Demand}

\subsection{Future Prediction}
The IEA (International Energy Agency) were in 2015 optimistic, predicting a decoupling of growing electricity generation from \cotwo emissions.
This was based on a prediction of low-carbon power generation.
Energy related \cotwo emissions are still predicted to grow because of fuels used outside electricity generation.
This IEA prediction is in line with the goal of limiting global temperature increase by $2\degree$ C, but to realistically achieve this would require unprecedented progress.
One of the major reasons for increased energy-related \cotwo emissions are growing demands for transport.
The global vehicle fleet is expected to double until 2035, maily becuase of increased ownership in emerging economies.

\subsection{World Electricity Consumption}
The world's electricity consumption has steadily increased even throughout global crises.
In later years the consumption in Europe and North America has not increased much, but a large increase in consumption can be observed for Asia.
It is clear that modernization of economies come with a rapid increase in electricity consumption.

\subsubsection{OECD and Non-OECD}
OECD (Organisation for Economic Co-operation and Development) is an organisation including 32 high income countries.
The primary energy consumption of OECD countries is not projected to increase substantially.
A large increase is however predicted for Non-OECD countries, mainly in China and India.
The projected increase of primary energy consumption in Non-OECD countries is dominated by Asia.
Some increase is predicted for countries in the middle east and Americas.
The predicted increase for Africa is very low.\\

The fastest growing fuels are renewables.
Use of oil, coal and gas is still predicted to grow in absolute value to cover up for the total demand.

\subsubsection{Driving Forces Behind Energy Demand}
The two driving forces behind increasing energy demand are

\begin{itemize}
    \item Increasing population
    \item Income Growth
\end{itemize}

Increased population and income throughout the last 100 years has resulted in the increased energy demand.
The global trend is increasing income, especially in low and medium income economies.
Population growth is however generally trending down.

\subsection{Economic Growth and Energy Concumption}
The concept of economic growth includes

\begin{itemize}
    \item Higher life expectancy
    \item Better health
    \item Better education
    \item Political stability
    \item ...
\end{itemize}

There is an almost linear relationship between increase in energy consumption and \cotwo emissions.
Economic growth is also closely tied to increasing energy consumption.
For a sustainable development there is a need to decouple economic growth from energy consumption and \cotwo emissions.
In the EU this decoupling has to some extent already taken place.
A similair decoupling is predicted in China.
The global electricity use grows faster than other energy, but the electricity use grows slower than the world's GDP.


