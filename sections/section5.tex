\section{Smart Meters}
Smart meters measure household electricity usage (sometimes also gas and water).
They digitally communicate these measurements to the utility provider and/or the consumer.\\

\subsection{Smart Meter Properties}
Smart meters can be directly connected to the utility, allowing the consumer access to the information only through the utilities servers.
Smart meters can also be WiFi-enabled, allowing direct interaction with the consumer, keeping the utility out of the loop.
This leads to higher privacy and often more granularity in information.
Note however that data from only one smart meter can make the consumer miss the larger picture. A utility can combine data in useful ways.\\

Smart meters enable
\begin{itemize}
    \item Remote reading
    \item Dynamic prizing
    \item Demand side load management
    \item New services
\end{itemize}

\subsubsection{Smart Meter Features}
Smart meters can measure voltage, current, phase angle and more.
Smart meters communicate with different frequencies (everything from daily to every second).
They can have support for 2-way communication.
Different communication interfaces exist, including WiFi, DSL and the mobile network.

\subsubsection{Challenges}
Smart meters are somewhat costly ($\sim$ 100 EUR) and come with installation costs.
It is unclear if the consumer or utility should pay.
For the utility deployment of smart meters might also come with costly changes in their backend systems.
It is also somewhat unclear what the business case for smart meters are
(Do utilities truly save money?, Can consumers really save on their electricty bill?).
There are privacy concerns about smart meters from consumers.
Many consumers also perceive the user interfaces as poorly designed and too technical.\\

In order to fully utilize smart meters there is a need for standardized interfaces to other home automation appliances.
There exists such an attempt to create a European standard for smart meters.\\

Pilot projects have shown that there are some easy savings in electrical energy possible from change in consumer behavior.
However only a few people are interested and they quickly lose the interest.

\subsubsection{Consumer benefit}
With savings estimated by pilot projects the average (German) household would hardly save any money by installing a smart meter. The small savings are eaten up by the installation cost.
Greater savings could be achieved in many households.
This motivates targeting households with larger electricity consumption.

\subsubsection{Smart Meters in Europe}
Currently large deployment of smart meters have been performed in Sweden and Italy.
In Sweden this was mandated by a law mandating monthly electricity bills.
Utilities in Sweden actually saved money by upgrading to smart meters simply because of the reduced costs in customer service call centers.
In Italy the deployment was motivated by need from utilities to limit consumer consumption for balancing the grid.
EU have directives pushing countries to investigate and deploy smart meters.

\subsubsection{Smart Meters as Enablers}
Smart meters enable
\begin{enumerate}
    \item \textbf{System efficiency} of the entire electricity system through demand response, coordination of consumption with clean energy production etc.
    \item \textbf{Energy Savings} through changes in consumer behavior motivated by feedback from smart meter enabled systems.
\end{enumerate}

It is important to keep in mind that deployment of smart meters achieves nothing in itself.
Energy and cost savings are only achieved by the programmes and structures that are enabled by the platform smart meters provide.

\subsection{Acceptance issues}

\subsubsection{Privacy}
Smart meters with high granularity create data from which everyday habits easily could be extracted.
The privacy issues are crucial for the acceptance of smart meters.
Smart meters that enable remote control also make consumers lose control.
Both technical and legal solutions are possible.
Consumers can be given options to opt in and out of features.
Communication can be encrypted to prevent adversaries from recovering sensitive information.
The data that leaves the household could be minimal, even though large amounts of data is collected in smart meter itself to enable functionality for the consumer.\\

Movements condemning smart meters exist, mainly in the US.
Some of the arguments of these groups seem based in pseudo-science and/or extremist beliefs.

\subsection{Smart Metering for Energy Conservation}
Without smart meters it is almost impossible for consumers to understand when and where the electricity is used. Smart meters enable consumers to decrease their consumption.
Smart meter data can be visualized through
\begin{itemize}
    \item Web Portals
    \item In-home Displays
    \item Smart Phone Apps
\end{itemize}

\begin{tcolorbox}
\textbf{Issues in Studies on Smart Metering}\\
Early studies on smart metering showed very promising results, but later studies show substantially lower savings.
Generally the savings reported decrease as sample size increases.
Common problems are

\begin{labeling}{\textbf{Allocation concealment bias}}
    \item [\textbf{Volunteer selection bias}]
    participants different than study's intended population
    \item [\textbf{Intervention selection bias}]
    participants choose their treatment group
    \item [\textbf{Sequence generation bias}]
    participants assigned to groups sequentially, not random
    \item [\textbf{Allocation concealment bias}]
    participants or researchers manipulate participants group assignment
    \item [\textbf{Blinding bias}]
    researchers intervene more with non-control groups
    \item [\textbf{Attrition bias}]
    participants withdraw from study because of assigned group, resulting in skewed statistics
\end{labeling}

 Most of these biases are avoided by using proper \textbf{Randomized Controlled Trials} (RCTs). Smart metering studies implemented as RCTs typically report savings of 1-3\%.

\end{tcolorbox}

