\section{Smart Grid Security}
Electricity is a key resource and therefore the grid is part of critical infrastructure.
Energy security typically has reffered to geopolitics, accidents and natural disasters.
Energy security is increasingly a matter if cybersecurity.

\subsection{Critical Infrastructure}
Critical infrastructure is infrastructure that is a critical enabler of economic activity and social welfare.
Economies tend to become increasingly specialized and rely more on critical infastructure for key resources and services.\\

The electrical grid has evolved over a long time and therefore contains of a mix of old and new technology.
The electrical grid is a patchwork of both physical and organisational networks.
It is highly dynamic and mutually interdependent with other critical infrastructure (in particular ICT).
This creates risk for cascading effects if one piece of critical infrastructure should break down.
Cyber-attacks against critical infrastructure has been proven to be a real threat.

\subsection{Fragility of Electricity Grids}
The implementation of smart grids potentially increase reliability thanks to fault detection and isolation.
Islanding (isolating self sustaining portions of the grid from problems) can make the grid more robust.
These improvements do however come at a cost of increased complexity and a larger attack surface for cyber attacks.\\

The physcial properties of electricity make the grid particularly hard to control.
Electricity travels at the speed of light, making it impossible for human decision makers to react to problem.
Electricity flows in all paths available and can not easily be stopped with some valve or switch.\\

\begin{tcolorbox}
    \textbf{Blackouts}
    \begin{labeling}{\textbf{Germany, 2006}}
    \item [\textbf{US, 2003}]
        A tree flashover caused a domino effect.
        Operators did not notice that power lines went out of service because of bug (race condition) in monitoring software.
    \item [\textbf{Italy, 2003}]
        Tree flashover in Switzerland caused blackout of nearly all of Italy through domino effect.
    \item [\textbf{Germany, 2006}]
    Scheduled shutdown of a line to allow a crusie ship safe passage result in unforseen cascade of line overloads.
        Europe was split in separate areas to stop cascade.
        Luckily automatic countermeasured managed to rebalance the grid.
    \end{labeling}

Blackouts of the electrical grids in all cases spilled over and severly crippled other critical infrastructure.
All these failures show that local problems quickly result in far-reaching consequences.
\end{tcolorbox}

\subsection{Cyber Attacks Against the Electrical grid}
The smart grid presents large challenges in cyber security.
The industry trend is to replace specialized software and hardware with commercial off-the-shelf systems using general internet based technology.
Physical security is an issue, especially considering the locations of much of the smart grid technology.
Maintaining high security as grids are getting smarter proves to be a challenge since most utilities do not have much experience with cyber security.\\

Some properties and changes of the smart grid that increas the attack surface are
\begin{itemize}
    \item Legacy systems
    \item Increased interconnection of systems
    \item New 2-way systems
    \item New customer touch points
    \item More subsystems, larger code base
\end{itemize}

\subsubsection{Examples of Cyber Attacks Against Electricity Infrastructure}
\begin{labeling}{\textbf{SQL Slammer Worm}}
    \item [\textbf{SQL Slammer Worm}]
    Worm that infected US nuclear power plant in 2003.
    Spread through business network to control network, eventually causing a denial of service of the control system.
    Fortunately did not affect analog backup system.
    \item [\textbf{Stuxnet}]
    Highly specialized computer worm infecting Windows systems. Carries malware payload specifically targeted at Siemens control systems.
    The target of Stuxnet was the Iranian uranium enrichment infrastructure.
    It is largely believed that the worm was created by US and Israelian cyber warfare institutions.
\end{labeling}
