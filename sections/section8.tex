\section{Smart Heating}
Space heating takes up a very large part of the households energy.
This motivates heating as a prime candidate for applying smart technology.
Space heating is becoming more effective, but the floor area in houses is increasing, making the total energy used for heating stay roughly constant.

\subsection{Factors Impacting Heating}
Factors that impact energy use for heating usually fall in the categories

\begin{itemize}
    \item Building
    \item Heating system
    \item User behavior
    \item Environment
\end{itemize}

\subsection{Thermostats}
Traditional thermostats need to be manually controlled.
The thermostat works as a simple control loop.
Saving energy in this setting would require turning down the heating when leaving home and then coming home to a cold home.
Keeping the temperature lower when away from home saves energy since the heating loss depends on the difference between indoors and outdoors temperatures. \\

\subsubsection{Programmable Thermostats}
Programmable thermostats can keep schedules of different temperatures.
They do require the users to program the schedules to match their daily routine.
Studies have shown that programmable thermostats rarely lead to savings.
Many feel overwhelmed by the programming and simply overrides the features.
Many do not have typical 9-5 jobs, making any programmed schedule unsuitable.

\subsubsection{Self Learning Thermostats}
With self learning thermostats sensors and learning algorithms replace the human in the loop.
The most well known self learning thermostat is the nest.
The nest has a one week intial learning period.\\

Users report a disconnect between their intentions and the heating decisons of self learning thermostats.
Users do not understand why the thermostat takes decisions and feel no way to communicate this to the system.
There is a problem of interpretability, explainability and user experience.

\subsection{Occupancy Sensing and Prediction}
Large savings as well as great comfort could be expected if the heating system knew perfectly where the residents are at all times.
This would require some combination of sensing presence, prediction schedules and adapting to changes in such schedules.

\subsubsection{Occupancy Prediction algorithms}
Prediction is key since heating need to start some time before residents arrive home.
Evaluation of occupancy prediciton algorithms can be done in comparison to a perfect oracle.
The practically achievable best prediction is however worse than any oracle.
The best achievable accuracy is 90\% and many existing algorithms reach an accuracy of 85\%.
Handling of exceptions in these algorithms can be done more or less automatic with sensors or user interaction.

\subsubsection{Conditions Impacting Savings}
Results from studies using occupancy prediction report widely varying energy savings.
Some factors that impact the savings achieved are

\begin{itemize}
    \item Accuracy of prediction
    \item Setback temperature
    \item Power of the heating system
    \item Duration of absence
    \item Building properties (insulation etc.)
    \item Outdoor temperature
\end{itemize}

\subsubsection{Occupancy Sensing}
Sensing occupancy can be done through sensors in the home, such as cameras or motion sensors.
It can also be done using sensors on humans, mainly through reporting of GPS position from a smartphone.
Another possible indicator of occupancy is the electricity consumption.
Extracting an occupancy sequence from electricity consumption data requires use of supervised or unsupervised learning algorithms.
Labeling of such a dataset is difficult for large datasets, making unsupervised or semi-supervised approaches preferable.

\subsection{Estimating Savings Potential of Smart Heating}
In order to simulate heating of buildings one has to create a thermal building model.
This model describes the heat conductivity between the property and environment.
Using such a model one can evaluate which households would achieve substantial savings by using smart heating.
This can then motivate targeting of policy efforts for energy saving.



