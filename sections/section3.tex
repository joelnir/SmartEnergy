\section{Electricity Generation, Cost and Storage}

Electricity is a versatile energy carrier.
The share of electricity in the worlds final energy consumption has been growing steadily since the 70s.
The increased share of electricity is tightly tied to economic growth.
One reason behind this is a shift in consumer preference since electricity feels like a clean fuel as it is being used.
Still over 1 billion people lack acces to electricity, mainly in Africa, India and parts of developing Asia.

\subsection{Energy Conversion Efficiency}
The energy conversion efficiency is the ratio between useful output of an energy conversion machine and the input.
For example this value is usually around 33\% for the coal-to-electricity process.
Losses occur both in processing (typically in the form of heat) and in transmission.\\

Consider the difference kinds of energy in a conversion and distribution process

\begin{labeling}{\textbf{Effective (Net) energy}}
    \item [\textbf{Primary energy}] goes into the system, e.g. crude oil
    \item [\textbf{Secondary energy}] intermediate type of storage, e.g. fuels for cars
    \item [\textbf{Effective (Net) energy}] final used energy, e.g. lighting
\end{labeling}

\subsection{Fuel Types in Electricity Production}

The worlds yearly electricty production was $\sim$ 25 000 TWh.
The dominating fuels for electricity production worldwide are still coal, gas and oil.
Various fuel types have different properties concerning

\begin{itemize}
    \item Cost
    \item Availability
    \item Emissions
    \item Other side effects
\end{itemize}

\subsubsection{Coal}
Coal represents $\sim$ 40\% of the worlds primary electric energy.
It has fairly low direct cost, but future costs because of environmental consequences are hard to predict.
Burning of coal emitts 700-800 g of \cotwo / kWh. Coal can also contain high amounts of sulfur and uranium which hurt the envinronment. Efficency of coal electricity generation is usually around 30\% with some modern approaches reaching as high as 45\%.
Coal has multiple other negative side effects.
Coal production is dangerous and cause thounsands of deaths each year.
Coal-fired powerplant is believed to cause thousands of premature deaths because of negative impacts on human health.
The RPR of coal is estimated to 90-180 years.

\begin{tcolorbox}
    \textbf{Reserves-to-production Ratio (RPR)}\\
    $$
    \text{RPR} = \frac{\text{Reserve}}{\text{Production}}
    $$
    The reserve is the amount of a resource known to exist and be economically recoverable.
    The production is the amount of a resource used in one year at the current rate.
    Note that RPR can be a poor predictor since production can change rapidly and improved mining technology can change the reserve as more of a resource is feasible to extract.
\end{tcolorbox}

\subsubsection{Oil}
Oil currently stands for a very small share of primary electric energy.
Oil has a very volatile cost (consider 1973 oil crisis etc.).
The efficiency of electricty generation using oil is similair to coal at around 30\%.
Burning oil emitts large amounts of \cotwo.
Other side effects of oil is the environmental danger of oil spills as well as geopolitical issues.
The RPR of oil is 17-50 years.

\subsubsection{Gas}
Gas today represents $\sim$ 22\% of electricity primary energy, but the share is increasing.
The price for gas is quite volatile.
Emissions from burning gas are 400-550 g \cotwo / kWh, but there are fewer other emissions compared to e.g. coal.
The efficiency of gas for electricity generation can reach up to 60\%.
Gas turbines are quick to ramp up and therefore suitable for peak load generation.
An issue with gas is that most exporting countries are politically instable.
The RPR of gas is 32-60 years.

\subsubsection{Hydropower}
Without pump storage hydropower's share of primary electric energy is $\sim$ 16\%.
Hydro is a sustainable source of energy with a cost of 5-10 cents / kWh.
The emissions are very low and only related to the construction of the plant.
Water turbines achieve 90\% efficiency.
Geographically potential cites for hydropowerplants are very unequally distributed.
A substantial fraction of the worlds potential hydropower is already today realized.
The construction of hydropowerplants does come with some environmental issues for the surrounding ecosystem.

\subsubsection{Nuclear}
Nuclear represents $\sim$ 11\% of world primary electrical energy. The cost for nuclear generated electricity follows that of coal and hydro, but can be highly impacted by politics.
The efficiency in electricity production for nuclear power is $\sim$ 44\%.
With the limited supply of uranium nuclear power has a RPR of 30-45 years.
Side effects of nuclear power are highly controversial. Accidents such as the Chernobyl and Fukushima accidents have fueled public backlash. The final storage of radioactive waste is also a considerable problem.

\subsubsection{Wind}
Wind power 2017 produced $\sim$ 4.4\% of the worlds electricity.
The share of wind power is however very different in different countries, with larger share in for example Denmark and Germany.
The cost of electricity from wind power is 5-10 cent / kWh for modern facilities.
The efficiency of wind turbines is in practice 40-47\%.
One issue wit wind power is that the generation is stochastic (not reliable, highly dependent on current weather).
Construction of new wind farms often encounter acceptance issues and the turbines themselves require large amounts of rare materials.
Wind turbines are claimed to effect the surrounding environment negatively, but this effect is often questioned.\\

The UK are in a privileged postition to deploy wind power because of very high wind speeds.
Wind power is highly volatile and therefore depends on good forecasts of wind speeds.
There is a belief that development in big data, machine learning and artificial intelligence will improve these forecasts and enable increased integration of renewable energy in the power grid.
Currently China and the US have the largest capactity for wind power (capacity here refers to generation under idela conditions).
The technological development of wind turbines has led to larger constructions with rotors with diameters over 100 m.

\subsubsection{Solar power (Photovoltaic, PV)}
PV today represents $\sim$ 2\% of electricity production to a cost of 10-25 cents / kWh.
The practical efficiency of photovoltaic is typically 16\%, but the theoretical limit is as high as 90\%.
The earth receives plenty more of energy from the sun than is needed. If we covered the earths surface with PV it would generate 5000 times the worlds energy demand.
Construction of solar cells require some hard to recover materials (also geopolitical issues).
As with wind power solar power is highly volatile and not just with the weather, but also with time of day and time of year.\\

The price of photovoltaic cells has decreased substantially since it's invention and in particular in the last 10 years. This trend is projected to continue and the prize fall another 50\% the next 10 years.
Since PV rely on the sun it's expected effectiveness is very different in different countries.
An alternative to photovoltaics is concentrating solar power, where mirrors capture sunlight and the heat drives steam turbines. This technology is still in a stage of engineering prototypes.

\subsubsection{Other renewables}
Biomass is a poor choice for electricity generation because of it's low efficiency levels. Large scale use could also negatively impact local food supplies.\\

Geothermal energy is today used for heating.
It is a reliable resource but drilling is costly and comes with safety issues (fracking, land stability).
There is also some \cotwo emissions expected from the fluids deep inside earth.\\

Tidal energy is more predictive than solar and wind power. There are however big concerns for the ecosystem with deploying tidal turbines.

\subsubsection{Summary of Fuel Types}
Table \ref{tab:summary_fuel} shows a summary of properties of different fuel types used in electricity generation.

\begin{sidewaystable}
\centering
\begin{tabular}{| l | c | c | c | c | c | c |}
    \hline
    \textbf{Fuel type} & \textbf{Share} & \makecell{\textbf{Cost}\\ \textbf{(cent/kWH)}} & \textbf{Efficiency} & \makecell{\textbf{Emissions}\\ \textbf{(g \cotwo/kWh)}} & \textbf{RPR (years)} & \textbf{Side Effects} \\ \hline
    Coal & 40\% & 5.2 & 30-45\% & \makecell{700-800\\+sulfur, uranium} & 90-180 & \makecell{Dangerous to obtain,\\ Negative health effects} \\ \hline
    Oil & 4\% & Volatile & 35\% & Plenty & 17-50 & \makecell{Oil spills, \\ Geopolitical Issues}\\ \hline
    Gas & 22\% & Volatile & 40-60\% & 400-550 & 32-60 & \makecell{Politically instable\\ exporters}\\ \hline
    Hydro & 16\% & 5-10 & 90\% & - & - & \makecell{Mostly realized,\\ Environmetnal issues}\\ \hline
    Nuclear & 11\% & 5 & 44\% & - & 30-45 & \makecell{Devastating accidents, \\ Nuclear waste}\\ \hline
    Wind & 5\% & 5-10 & 40-47\% & - & - & \makecell{Volatile generation, \\ Construction materials}\\ \hline
    Solar & 2\% & 10-25 & 16\% & - & - & \makecell{Volatile generation, \\Construction materials}\\ \hline
\end{tabular}

\caption{Summary of properties of different fuel types}
\label{tab:summary_fuel}
\end{sidewaystable}

\subsubsection{Future of renewables}
In order to change completely to renewable sources for electricity production would require more storage to tackle volatility. One main issues are long and short term storage without losses.
To replace other energy with electricity from renewables would also require further development in using electricity for the transport sector.
There is an increased need for a smarter power grid.

\subsection{Cost Factors of Electricity Generation}

Cost factors of any electricity generating system include

\begin{itemize}
    \item Initial investment
    \item Capital and discount rate
    \item Fuel
    \item Operation and Maintenance
\end{itemize}

\subsubsection{Surrounding Costs}
There are usually also surrounding costs that are rarely considered. These include distribution costs, waste disposal, costs related to environmental impact etc.
These costs are not reflected in the electricity prize, but are rather costs that society as a whole must bear. Coals tops the charts on these surrounding costs, mainly because of it's negative health impacts on humans.\\

\subsubsection{LCOE}
The LCOE (Levelized Cost of Energy) is the prize at which electricity must be generated for the project to break even over it's lifetime. The LCOE of renewable electricity generation can be much higher than that of genration based on fossil fuels. The cost of renewables is dominated by capital cost (initial investment) whereas for non-renewable generation fuel dominates the costs.

\subsubsection{Retail Prize}
Households and industry pay different prices for electricity. Industry typically have politically subsided prize, fininzed by a higher prize for consumers.
The retail prize of electricity can be broken down as
$$
\text{prize} = \text{Generation} + \text{Transmissions \& Distribution} + \text{Fees} + \text{Taxes}
$$
This means that generation is typically a only a smaller part of the retail prize.
In Switzerland the prize is dominated by distribution and generation. The electricity prize can vary a lot between countries, for example with a factor 2 between France and Germany.

\subsection{Electrical Energy Storage (EES)}
Electricity is typically hard to store. EES can help with
\begin{itemize}
    \item Increasing the efficiecny of the power system
    \item Improve grid stability and reliability
    \item Increase energy security
\end{itemize}
Considering the fluctuations in generation based on renewable energy sources EES looks to be a neccesity in the future.
EES can help balance supply and demand to handle peak loads.
Storage allows for local buffers of electricity for later use.

\subsubsection{Types of EES}
The most common technologies for storing electricty are

\begin{itemize}
    \item Mechanical / Kinetic
        \begin{itemize}
            \item Pumped hydro
            \item Compressed air storage
            \item Flywheels
        \end{itemize}
    \item Electric / Magnetic field
        \begin{itemize}
            \item Supercapacitors
            \item Superconducting magnets
        \end{itemize}
    \item Power to gas
    \item Batteries
\end{itemize}

The mechanical technologies (with the exception of flywheels) are typically more suitable for long-term, high energy storage and the electro-magnetic technologies more sutiable for short-term, high power sotrage.
Different battery technologies are more or less suitable for different power and energy levels.

\subsubsection{Performance Characteristics of EES}
The main considerations of EES are
\begin{itemize}
    \item Energy rating
    \item Power rating
    \item Response time
    \item Self-discharge rate
    \item Lifetime
    \item Cost
    \item Environmental impact
    \item Spatial requirements
    \item Hazards
\end{itemize}

\subsubsection{Storage in the Grid Today}
99\% of the storage capacity today is pumped hydro.
In Europe the potential for stored hydropower is largely already realized.
The efficiency of this entire pumping and generation process is 70-85\%.






