\section{Electricity Generation, Cost and Storage}

Electricity is a versatile energy carrier.
The share of electricity in the worlds final energy consumption has been growing steadily since the 70s.
The increased share of electricity is tightly tied to economic growth.
One reason behind this is a shift in consumer preference since electricity feels like a clean fuel as it is being used.
Still over 1 billion people lack acces to electricity, mainly in Africa, India and parts of developing Asia.

\subsection{Energy Conversion Efficiency}
The energy conversion efficiency is the ratio between useful output of an energy conversion machine and the input.
For example this value is usually around 33\% for the coal-to-electricity process.
Losses occur both in processing (typically in the form of heat) and in transmission.\\

Consider the difference kinds of energy in a conversion and distribution process

\begin{labeling}{\textbf{Effective (Net) energy}}
    \item [\textbf{Primary energy}] goes into the system, e.g. crude oil
    \item [\textbf{Secondary energy}] intermediate type of storage, e.g. fuels for cars
    \item [\textbf{Effective (Net) energy}] final used energy, e.g. lighting
\end{labeling}

\subsection{Fuel Types in Electricity Production}

The worlds yearly electricty production was $\sim$ 25 000 TWh.
The dominating fuels for electricity production worldwide are still coal, gas and oil.
Various fuel types have different properties concerning

\begin{itemize}
    \item Cost
    \item Availability
    \item Emissions
    \item Other side effects
\end{itemize}

\subsubsection{Coal}
Coal represents $\sim$ 40\% of the worlds primary electric energy.
It has fairly low direct cost, but future costs because of environmental consequences are hard to predict.
Burning of coal emitts 700-800 g of \cotwo / kWh. Coal can also contain high amounts of sulfur and uranium which hurt the envinronment. Efficency of coal electricity generation is usually around 30\% with some modern approaches reaching as high as 45\%.
Coal has multiple other negative side effects.
Coal production is dangerous and cause thounsands of deaths each year.
Coal-fired powerplant is believed to cause thousands of premature deaths because of negative impacts on human health.
The RPR of coal is estimated to 90-180 years.

\begin{tcolorbox}
    \textbf{Reserves-to-production Ratio (RPR)}\\
    $$
    \text{RPR} = \frac{\text{Reserve}}{\text{Production}}
    $$
    The reserve is the amount of a resource known to exist and be economically recoverable.
    The production is the amount of a resource used in one year at the current rate.
    Note that RPR can be a poor predictor since production can change rapidly and improved mining technology can change the reserve as more of a resource is feasible to extract.
\end{tcolorbox}

\subsubsection{Oil}
Oil currently stands for a very small share of primary electric energy.
Oil has a very volatile cost (consider 1973 oil crisis etc.).
The efficiency of electricty generation using oil is similair to coal at around 30\%.
Burning oil emitts large amounts of \cotwo.
Other side effects of oil is the environmental danger of oil spills as well as geopolitical issues.
The RPR of oil is 17-50 years.

\subsubsection{Gas}
Gas today represents $\sim$ 22\% of electricity primary energy, but the share is increasing.
The price for gas is quite volatile.
Emissions from burning gas are 400-550 g \cotwo / kWh, but there are fewer other emissions compared to e.g. coal.
The efficiency of gas for electricity generation can reach up to 60\%.
Gas turbines are quick to ramp up and therefore suitable for peak load generation.
An issue with gas is that most exporting countries are politically instable.
The RPR of gas is 32-60 years.

\subsubsection{Hydropower}
Without pump storage hydropower's share of primary electric energy is $\sim$ 16\%.
Hydro is a sustainable source of energy with a cost of 5-10 cents / kWh.
The emissions are very low and only related to the construction of the plant.
Water turbines achieve 90\% efficiency.
Geographically potential cites for hydropowerplants are very unequally distributed.
A substantial fraction of the worlds potential hydropower is already today realized.
The construction of hydropowerplants does come with some environmental issues for the surrounding ecosystem.

\subsubsection{Summary of Fuel Types}

\begin{sidewaystable}

    \centering
\begin{tabular}{| l | c | c | c | c | c | c |}
    \hline
    \textbf{Fuel type} & \textbf{Share} & \makecell{\textbf{Cost}\\ \textbf{(cent/kWH)}} & \textbf{Efficiency} & \makecell{\textbf{Emissions}\\ \textbf{(g \cotwo/kWh)}} & \textbf{RPR (years)} & \textbf{Side Effects} \\ \hline
    Coal & 40\% & 5.2 & 30-45\% & \makecell{700-800\\+sulfur, uranium} & 90-180 & \makecell{dangerous to obtain,\\ negative health effects} \\ \hline
    Oil & 4\% & Volatile & 35\% & Plenty & 17-50 & \makecell{Oil spills, \\ Geopolitical Issues}\\ \hline
    Gas & 22\% & Volatile & 40-60\% & 400-550 & 32-60 & \makecell{Politically instable\\ exporters}\\ \hline
    Hydro & 16\% & 5-10 & 90\% & Negligible & - & \makecell{Mostly realized,\\ Environmetnal issues}\\ \hline

    \hline
\end{tabular}

\caption{Summary of properties of different fuel types}
\end{sidewaystable}


