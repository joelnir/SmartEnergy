\section{The Energy Consumption of ICT}
Since the early days of computing the energy conspumption of ICT has been discussed.
It is generally a topic with little consensus and widely varying estimates of emissions  attributed to computing.

\subsection{Network Technology}
Computer networks can be divided into multiple stages betweem the backbone of the internet and consumer devices.
Close to the consumers in the networks are Digital Subscriber Line Access Multiplexers (DSLAM).
These are local nodes providing neighborhoods with internet connection.
In the backbone of the network are poweful core internet routers.
Network-related devices are typically split into the three categories: User, Network and Data centre.

\subsection{Data Centers (DCs)}
Data centers are typically located in cold areas, odten with closevy supply to water.
Large data centers have their own energy supply station.

\subsubsection{PUE}
The Power Usage Efficiency (PUE) is a measure of efficiency of a data center.
It can be calculated as
$$
\text{PUE} = \frac{\text{Energy consumption of entire DC}}{\text{Energy for IT equipment}}
= 1 + \frac{\text{Non-IT energy}}{\text{IT Energy}}
$$
The non-IT energy consist mainly of cooling. Some energy is spent on lightning and some is lost in transformation.
The average PUE values improved substantially until 2011.
The biggest infrastrucutre efficiency gains happened 5 years ago.
Further improvement require significant investments.
Google report PUEs as low as 1.12.

\subsubsection{Cooling}
Data centers are generally air cooled through a Computer Room Air Conditioning Unit (CRAC).
DCs are usually built with a raised floot to allow pumping of cols air under the floor.
A more modern layout is alternating cold and hot aisles between servers.
Some alternative concepts that are still in the research stage are server immersion cooling and on-chip eater cooling.
Efforts exist to reuse heat from DC in district heating.

\subsubsection{Servers in the world}
About half of servers are in data centers.
The other half are directly at enterprises.
Servers can be estimated responsible for 200-300 TWh/year.
This represents less than 2\% of worldwide electricity consumption.

\subsection{The Network}
The network can be modelled top down (start by estimating equipment) or bottom up (start by estimating users and usage).
Top down models tend to overstate consumption whereas bottom up models tend to understate consumption.
It is often somewhat unclear what equipment should be considered part of the internet.\\

Energy intensity of equipment close to customers is usually more time dependent.
Energy intensity of equipment deeper in the network depends more on traffic.

\subsubsection{Network Growth}
The data sent over the internet has grown rapidly and continues to do so.
In particular global mobile traffic and machine-to-machine traffic is growing with high rates.
The energy required by data centers is estimated to increase substantially in order to handle this increased traffic.
The increase in energy efficiency is lower than the growth of traffic.

\subsection{Future Predictions}
The overall wroldwide electricity consumption of PCs has increased.
The growth has been (and continues to be) driven by

\begin{itemize}
    \item Higher penetration ates
    \item Increased hours of use
    \item More powerful PCs
    \item Larger screens
\end{itemize}

In households in mature markets low-power devices such as tablets and smartphones are substituting PCs and laptops.
For these low-power devices a larger fraction of their environmental impact comes from production (instead of use).
Functional convergence also decrease amount of devices needed.




